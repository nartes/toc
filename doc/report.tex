\documentclass{amsart}

\usepackage[utf8]{inputenc}
\usepackage[russianb]{babel}
\usepackage{moreverb}
\usepackage{amssymb}
\usepackage{multicol}

\begin{document}
\section{Задачи}

\subsection{5}

Исходное условие:
Дан граф $G'=(V',E'), V'=\{1,\dots,k\}, |E'|=m$.
Необходимо распознать наличие клики размера $c$.

Сводим к задаче
$P1 | Prec, w_j= \{ 1,2,\frac{|A|}{2}-7 \}, pj=p | \sum w_j T_j$,
где $T_j = \max \{C_j - d_j, 0\}$.

\[
  v_i, \quad i=\overline{1,k}, \quad p_{v_i} = 1, \quad
  w_{v_i} = W, \quad d_{v_i} = k + l
\]

\[
  e_{ij}, \quad (i,j) \in E, \quad p_{e_{ij}} = 1, \quad
  w_{e_{ij}} = 1, \quad d_{e_{ij}} = c + l
\]

\[
  Prec: v_i \to e_{ij}, \quad v_j \to e_{ij}, \quad
  y = ?
\]

Если существует клика размера $c$, то

\begin{verbatim}
[         c           ][            l         ][             ]
     вершины клики            ребра клики           ???
\end{verbatim}

\[
  \sum w_j T_j =
  \sum_{<V>} t_i W +
  \sum_{<E_1>} t_i + \sum_{<E_2>} t_i,
\]
где $<E_1>$ множество ребер до $l+k$, а $<E_2>$ множество ребер
после $l+k$.

1. Простои только ухудшают целевую функцию.

2. Между $c+l$ и $k+l$ ребра предпочтительнее чем вершины, если
только не нарушается порядок предшествования.

Давайте поступим другим образом:

\[
  d_{v_j} = k + m, \quad w_{v_j} = 1
\]

\[
  \sum_{<E>} t_i \leqslant m - l
\]
то есть только ребра могут запаздывать

$\sum U_j$ новая целевая функция.
$\sum U_j = m - l$ есть условие задачи распознавания.
Если в графе нет клики, то упаковка по крайней мере $|E|=l$ ребер
на индуцированном графе $|V|=c$ вершинами не возможна, таким образом
нужна больше вершин, тогда по крайней мере $m - l +1$
работ запоздают.

\subsection{Лекция 11}
We have $M$ machines, and $n$ equal jobs to be processed.
Each machine can process only one job at a time and has $W_h$ is a price
for one time unit per $H=1,\dots,M$.

The processing time depends on $k$ -- a number of processed jobs and
$t$ -- a time of begin processing:
\[
  p_H(k,t) = \lambda_H(k-1) + \phi_H(k-1)t
\]
$\lambda_H \geqslant 0, \phi_H \geqslant 1$ are functions.

We are to find schedule $s$ so, that a cost function is minimized
\[
  \sum_{H=1}^M W_H
    \left(
      \sum_{k=1}^{n_H(s)} p_H(k,t_H^k(s))
    \right) \to \min
\]
$n_H(s)$ is the number of jobs per machine,
$t_H^k(s)$ -- start time of processing job $k$ at a machine $H$ and
equals:
\begin{align*}
  t_H^k(s)
    & \geqslant t_H^{k-1}(s) + p_H(k-1, t_H^{k-1}(s)) = \\
    & = \underset{>0}{\lambda_H(k-1)} +
      \underset{\geqslant 2}{(1 + \phi_H(k-1))}t_H^{k-1}(s)
\end{align*}

Let's look at
$p_H(k,t) = \lambda_H(k-1) + \underset{\geqslant 1}{\phi_H(k-1)}t$,
and come up with a conclusion
that thus the processing time
is lower bounded by geometrical progression with a divisor $\geqslant$ 2.

Let's make equality by adding $\Delta$:
\[
  t_H^k(s) + \Delta_H^k(s) =
  t_H^{k-1}(s) + p_H(k-1, t_H^{k-1}(s))
\]
Then we claim that $\Delta_H^k(s) = 0$, cause otherwise each additional
time unit gains equal unit to $p_H$, and thus $W_H$ units to cost
function, and at least 2 units to next start times on the machine.

Thus each schedule can specified by $n_H(s)$ per $H=1,\dots,M$.

Notable fact, that $p_H(k)$ is strict monotonuous function.

\begin{multicols}{2}
\[
  \begin{matrix}
    W_1 \times \\
    W_2 \times \\
    W_3 \times \\
    \dots \\
    W_M \times
  \end{matrix}
  \left\{
    \begin{matrix}
      p_1^1 & < & p_1^2 & < & p_1^3 & < & \dots \\
      p_2^1 & < & p_2^2 & < & p_2^3 & < & \dots \\
      p_3^1 & < & p_3^2 & < & p_3^3 & < & \dots \\
      & & & \dots \\
      p_M^1 & < & p_M^2 & < & p_M^3 & < & \dots \\
    \end{matrix}
  \right\}
\]

\hfill

On the left you see possible processing time for each machine.
We are to choose first $n_H$ at each machine.
Thus we can not skip anything. We can multiply each row by $W_H$
and inequalities remain.
\end{multicols}

Let us consider the following example:
\begin{multicols}{2}
\[
  \begin{matrix}
    a & 1 & 2 & 3 & 5 \\
    b & 5 & 6 & 7 & 10 \\
    c & 1 & 2 & 3 & 4 \\
    d & 4 & 8 & 12 & 20 \\
  \end{matrix}
\]

\hfill

\[
  n = 7
\]

If we are adding any sequence of $S_H^i$ then its first number
must be greater than any other not taken one, otherwise
optimization can occur
\[
  (S_H^1, S_H^2,\dots, S_H^l) \Rightarrow S_H^1 \geqslant \forall S_*^1
\]
\end{multicols}

Let's apply greedy algorithm and thus we obtain the following sequence
where wirst $n$ numbers are the answer for the original problem.
So first 7 numbers are the answer for the $n=7$.
\begin{equation}\label{E:greedy_seq}
  \begin{matrix}
    a_1 & c_1 & a_2 & c_2 & a_3 & c_3 & c_4 & d_1 & a_4 & \dots \\
    1,  & 1,  & 2,  & 2,  & 3,  & 3,  &4,  & 4,   & 5,  & \dots \\
  \end{matrix}
\end{equation}

We are to choose amount of first numbers at each row.
It can be coded as ${p_H^k | H = 1,\dots,M, k=1,\dots,n_H}$ -- set
of choosen numbers.

Let's define $s_0=()$. The next number, $s_1=(s_1^1)$
let's choose as the smallest possible.
As each row are ascending ordered, then our sequence
$s_n=(s_n^1, s_n^2, \dots, s_n^n)$, particularly \eqref{E:greedy_seq},
will be monotonously increasing.

Let's take any correct sequence $s'$ and map it to the $s_{Mn}$,
the sequence of $M \cdot n$ first numbers choosen by greedy algorithm.

The greatest number in $s'$ wiill are in a row, where it must be
the right most, according to asc order. So removing it from $s'$
will result in still correct sequence.

Then let's find the smallest not taken number in $s'$. It must be in
a certain row. The next number to the right most taken will be the
one we are searching for, again because of asc order. And as it is
next to the taken one then it can be taken and will result
in a correct sequence.

Take numbers are those that are present in $s'$ and are mapped as taken
in $s_{Mn}$.

It has sense till the number of the right most in $s_{Mn}$ taken number
has the position greater than $n$.

Because after that we have the correct sequence consisting of exactly
first $n$ numbers from $s_{Mn}$.

And it has the minimal possible total sum of numbers, as any other
replacement will only increase the total sum.

\subsection{Lecture's note 12}

\begin{multicols}{2}
  \hfill
  \begin{equation}\label{E:equal_lp}
    \begin{aligned}
      & \lambda_H(k) = \lambda(k) \\
      & \phi_H(k) = \phi(k)
    \end{aligned}
    \quad \forall H \in \{1,\dots,M\}
  \end{equation}
  \hfill
  \begin{equation}\label{E:W_sorted}
    W_1 \leqslant W_2 \leqslant \dots \leqslant W_M
  \end{equation}
  \hfill
\end{multicols}

\begin{equation}\label{E:pl_minmax}
  \begin{aligned}
    & \text{Moreover\quad }
    \forall H_1, H_2: W_{H_1} < W_{H_2} \Rightarrow \\
    & W_{H_2} < W_{H_1} (1 + \phi_{\min})
      \min\{
        \lambda_{\min} / \lambda_{\max},
        \phi_{\min} / \phi_{\max}
      \} \\
    & \text{or \quad}
    W_{H_2} < W_{H_1} \phi_{\min}
  \end{aligned}
\end{equation}

Above are the additional restritctions regarding original problem.
The problem remains the same, i.e. to minimize the total cost function.

We can apply the greedy algorithm here too. And taking into account
that \eqref{E:equal_lp} and \eqref{E:W_sorted} then also modifying
algorithm so that it chooses the lowst machine number in case of equal
additional price to cost function.

\textbf{To be proved:}
{\it And in the end we come up
with uniform distribution of tasks.}

At least we can say that
\[
  n_1(s^*) \geqslant n_2(s^*) \geqslant \dots \geqslant n_M(s^*),
\]
where $s^*$ -- the optimal solution.

Let's revise $p(k)$ and $L(k)$:
\[
  \begin{aligned}
    p(k) = \lambda(k-1) + \phi(k-1)L(k-1) & \\
    L(k) = \lambda(k-1) + (1 + \phi(k-1))L(k-1) &
  \end{aligned}
  \quad k > 0
\]

\begin{multicols}{2}
  \[
    \begin{array}{r|rrrrrr}
      \# & 0 & 1 & 2 & 3 & 4 & 5 \\
      \hline
      \phi:     & 1 & 2 & 1 & 1 & 3 & 4 \\
      \lambda:  & 1 & 1 & 2 & 3 & 1 & 2 \\
    \end{array}
  \]

  \[
    W_1 = 1, W_2 = 2, W_3 = 3
  \]

  \hfill

  \[
    \begin{array}{r|lr|r|r|r|}
      \hline
      \#1 & L   & p   & W_1p & W_2p & W_3p \\
      \hline
      0   & 0   & 0   & 0    & 0    & 0     \\
      1   & 1   & 1   & 1    & 2    & 3     \\
      2   & 4   & 3   & 3    & 6    & 9     \\
      3   & 10  & 6   & 6    & 12   & 18    \\
      4   & 23  & 13  & 13   & 26   & 39    \\
      5   & 93  & 70  & 70   & 140  & 210   \\
      6   & 467 & 374 & 374  & 748  & 1114  \\
      \hline
    \end{array}
  \]
\end{multicols}

The last three columns represent the possible additions to total
sum. And thus we are to choose how much we take at each column
in order to minimize total sum.

\[
  \begin{aligned}
    & W_1p_1 <W_1p_2 < W_1p_3 < \dots \\
    & W_2p_1 <W_2p_2 < W_2p_3 < \dots \\
    & W_3p_1 <W_3p_2 < W_3p_3 < \dots \\
    & \dots
  \end{aligned}
\]

\begin{equation}
  \begin{aligned}
    & W_{H_1} < W_{H_2} \Rightarrow
    & \left[
      \begin{aligned}
        & W_{H_2} < W_{H_1} (1 + \phi_{\min})
          \min\{
            \frac{\lambda_{\min}}
              {\lambda_{\max}},
            \frac{\phi_{\min}}
              {\phi_{\max}}
          \} \\
        & W_{H_2} < W_{H_1} \phi_{\min}
      \end{aligned}
    \right.
  \end{aligned}
\end{equation}

Let's take according to the greedy algorithm first few numbers, i.e.
$W_1p_1=1$, $W_2p_1=2$, $W_1p_2=3$, $W_3p_1=3$, $W_1p_3=6$.
Then it is easy to see that 6 on the third row is preferred over
the 9 on the second row.

\begin{center}$W_1p_3 < W_3p_2$ ?!, i.e. $6 < 9$.\end{center}

Let's checkout \eqref{E:pl_minmax} according this instance:
$3 < 1 \cdot (1 + 1) \min \{ \frac{1}{3}, \frac{1}{4} \}$.
So it is not true thus assertion failed about uniform assignment (schedule).

The following constraint
\begin{equation}\label{E:uni_con}
  W_1p_k \geqslant W_Mp_{k-1}, \forall k \geqslant 2
\end{equation}
holds iff a schedule has a uniform assignment. Of course with
assumption that $W_1 \leqslant W_2 \leqslant \dots \leqslant W_M$.

For the simples formulas argument $k$ of functions $p$, $L$, $\lambda$
and $\phi$ are moved to the lower index.

\[
  \begin{aligned}
    p_k & = \lambda_{k-1} + \phi_{k-1}L_{k-1} = \\
        & = \lambda_{k-1} + \phi_{k-1}
          ( \lambda_{k-2} + (1+\phi_{k-2})L_{k-2} )
  \end{aligned}
\]

\[
  p_{k-1} = \lambda_{k-2} + \phi_{k-2}L_{k-2}
\]

Let's make the constraint \eqref{E:uni_con} stronger:

\[
  \begin{aligned}
    & W_1(\lambda_{k-1} + \phi_{k-1}p_{k-1} + \phi_{k-1}L_{k-2}) \geqslant
        W_1\phi_{k-1}p_{k-1} \geqslant \\
    & W_1\phi_{\min}p_{k-1} \geqslant W_Mp_{k-1} \Rightarrow
  \end{aligned}
\]

\begin{equation}\label{E:str_sim_uni_con}
  W_1\phi_{\min} \geqslant W_M
\end{equation}

Thus \eqref{E:str_sim_uni_con} holds iff the schedule has uniform
distribution of jobs among machines.

Let's obtain yet one strong form of the constraint \eqref{E:uni_con}:

\[
  W_1(\lambda_{k-1} + \phi_{k-1}(\lambda_{k-2} + \phi_{k-2}L_{k-2})
    + \phi_{k-1}L_{k-2}) \geqslant
      W_M(\lambda_{k-2} + \phi_{k-2}L_{k-2})
\]

\[
  W_1(\lambda_{\min}(1+\phi_{\min}) + \phi_{\min}(1+\phi_{\min})L_{k-2})
    \geqslant W_M(\lambda_{\max} + \phi_{\max}L_{k-2})
\]

\begin{equation}\label{E:str_str_uni_con}
  \begin{cases}
    & W_1 \lambda_{\min}(1+\phi_{\min}) \geqslant W_M\lambda_{\max} \\
    & W_1 \phi_{\min}(1+\phi_{\min}) \geqslant W_M\phi_{\max}
  \end{cases}
\end{equation}

And finishing with the same conslusion, that \eqref{E:str_str_uni_con}
holds iff the schedule has uniform distribution of jobs among machines.

\#1 Let's consider the following subproblem ({\bf LastDigit}):
\[
  M=2, W_1=2, W_2=3, \lambda_H(k)=\lambda=\phi_H(k)=\phi(k), \forall k
\]
We are to find out whether $F(s^*) \equiv 5 (\bmod 10)$,
where $s^*$ is the optimal solution for the such problem.

\#2 Let's take the following instance of the partitioning problem:
\[
  e_1, e_2, \dots, e_p \text{\quad - the given set}
\]
And we are to find such $z \in { \{0,1\} }^p$, that
$\sum_{l=1}^p e_l z_l = E$, while $\sum_{l=1}^p e_l = 2E$.

Let's reduce the problem \#2 to the problem \#1.

For that let's define
\[
  \phi(k) =
  \begin{cases}
    4, \sum_{l=1}^p e_l b_l(k) \not = E, \\
    3, \sum_{l=1}^p e_l b_l(k) = E,
  \end{cases}
\]
where $b(k)$ is a binary representation vector for a number $k$.

Let's checkout uniform schedule constraints:
\[
  2 < 3 \text{ and } 2 (1 + 3) \frac{3}{4} = 6 > 3
  \Rightarrow \text{uniform schedule}
\]

Let's choose some values for $\phi(k)$:
\[
  \phi_k = (4, 4, 4, 3, 3, 4, 3, 4, 4).
\]

Let's revise the formulas of $p_k$ and $L_k$:
\[
  p_k=\lambda_{k-1} + \phi_{k-1}L_{k-1},
  L_k = \lambda_{k-1} + (1 + \phi_{k-1})L_{k-1}
\]

Let's calculate first few values for $p_k$ and $L_k$:
\[
  \begin{array}{r|ccccccccc}
    & 1 & 2 & 3 & 4 & 5 & 6 & 7 & 8 & 9 \\
    \hline
    p & 4 & 20 & 100 \\
    L & 4 & 24 & 124 & 499 & 1999 & 9999 & 39999 & 199999 & 999999 \\
  \end{array}
\]

Let's assume that there is no valid partition for the instance of the problem
\#2. Then $\lambda_k=\phi_k=4$.

\begin{align*}
  & T_{n+1} = 4 + 5 T_n \\
  & T_0 = 0 \\
  & T_n = 4 + 5 T_{n-1} \\
  & T_{n+1} = 4 + 20 + 25 T_{n-2}
    = 4 + 20 + 100 + 125 T_{n-2} \\
  & 4(1 + 5 + \dots + 5^{n-1}) + 4 \cdot 5^n=
    \frac{4(1-5^{n+1})}
      {1 - 5} = 5^{n+1} - 1 \Rightarrow
\end{align*}
\[
  \begin{cases}
    W_1L_k = 2 (5^{k+1}-1) \equiv 8 (\bmod 10), \\
    W_2L_k = 3 (5^{k+1}-1) \equiv 2 (\bmod 10),
  \end{cases}
\]
i.e. last digit in case of absence the solution for the partition problem
equals $F(s^*)=W_1L_{n_1(s^*)}+W_2L_{n_2(s^*)} \equiv 0 (\bmod 10)$.

Let's apply different $\phi_k$ to the last digit of $L_k$ when it equals 9:
\[
  9 \cdot 5 + 4 = 49, \quad 9 \cdot 4 + 3 = 19
\]
Thus at least one digit 9 is enough to make the rest sequence $L_k$ to containt
it.

Let's suppose that there at least one valid partition for the problem \#2.
Then there is such a $k \in \{1,\dots,2^p\}$ that $\phi_{k-1}=3$.
\begin{align*}
  & L_{k > 0}=(5^{k}-1) \equiv 4 (\bmod 10),
    \ 4 \cdot 4 + 3 = 19 \equiv 9 (\bmod 10) \\
  & L_0 = 0, \ 4 \cdot 0 + 3 = 3
\end{align*}
Thus if at least one $k$ satisfies partitioning
then $L_k \equiv 9 \lor 3 \ (\bmod 10)$.

Let's note that if $k=1$ represents a good partition then $L_1=3$.
It is easy to see that last digit 3 transforamtes in the following way:
\begin{align*}
  & L:\ 3\ 15\ 63 \Rightarrow 3 \to 5 \to 3 \\
  & L:\ 3\ 19\ 99 \Rightarrow 3 \to 9 \to 9
\end{align*}

So if there at least one good partition
then last digit of $L_{2^p}$ can be 3, 5 or 9.

{\it Let's finaly choose the value $n$ and it will be $n=2\cdot2^p$.}
Then the last digit of $F(s^*)$ might be:
\begin{align*}
  & (2 + 3) 9 = 45 \equiv 5\ (\bmod\ 10) \\
  & (2 + 3) 3 = 15 \equiv 5\ (\bmod\ 10) \\
  & (2 + 3) 5 = 25 \equiv 5\ (\bmod\ 10)
\end{align*}

So partition exists iff $F(s^*) \equiv 5\ (\bmod\ 10)$.

{\it
  Of course we rely upon the fact that the optimal schedule is uniform
  and so it contains $2^p$ jobs per each of two machines.
}

The input of both problem \#1 and \#2 is O(p). Any function $\lambda_k$
or $\phi_k$ can be calculated with O(p) time.
And as the problem \#1 in NP and we reduced the NP-c problem \#2 to it
with polynomial time then of course the problem \#1 is the NP-c.

\end{document}
